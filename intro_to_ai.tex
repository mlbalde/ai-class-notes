\documentclass[10pt,a4paper]{article}
\usepackage[utf8x]{inputenc}
\usepackage{ucs}
\usepackage{amsmath}
\usepackage{amsfonts}
\usepackage{amssymb}
\usepackage{rotating}
\usepackage{color}
\usepackage{algorithm}
\usepackage{algorithmic}

\newtheorem{defin}{Definition}
\newtheorem{ex}{Example}

% Draws a cylinder with "DB" in the middle
\newsavebox{\DB}
\savebox{\DB}{
\qbezier(0,30)(15,25)(30,30)
\qbezier(0,30)(15,35)(30,30)
\qbezier(0,5)(15,0)(30,5)
\put(0,5){\line(0,1){25}}
\put(30,5){\line(0,1){25}}
\put(6,12){\large DB}
}

% Draws a small stick figure
\newsavebox{\stickman}
\savebox{\stickman}{
\put(5,20){\circle{8}}
\put(5,16){\line(0,-1){12}}
\put(0,13){\line(1,0){10}}
}

% For drawing the box around nodes at the end of Section 2
\newsavebox{\nodeBox}
\savebox{\nodeBox}{
\put(0,0){\line(0,1){50}}
\put(0,0){\line(1,0){30}}
\put(30,50){\line(-1,0){30}}
\put(30,50){\line(0,-1){50}}
}

\newenvironment{itemize_packed}{
\begin{itemize}
\setlength{\itemsep}{0pt}
\setlength{\parskip}{0pt}
}{\end{itemize}}

\title{Notes for Introduction to Artificial Intelligence}
\begin{document}

%===============================================================================

\part{Welcome to AI}

\begin{defin}[Intelligent Agent]
An \emph{\textbf{Intelligent Agent}} observes the environment using its \emph{\textbf{sensors}}, makes a decision, and affects the environment using its \emph{\textbf{actuators}}.
\end{defin}

\begin{center}
\begin{picture}(200,100)
\thicklines
\put(20,20){\line(1,0){70}}
\put(90,20){\line(0,1){70}}
\put(20,20){\line(0,1){70}}
\put(20,90){\line(1,0){70}}
\put(40,92){Agent}
\put(170,55){\oval(40,90)}
\put(167,88){\begin{rotate}{-90}\large{Environment}\end{rotate}}
\put(158,90){\vector(-4,-1){80}}
\put(80,40){\vector(4,-1){80}}
\put(100,80){\begin{rotate}{15}Sensors\end{rotate}}
\put(100,40){\begin{rotate}{-15}Actuators\end{rotate}}
\color{red}
\put(65,70){\vector(0,-1){30}}
\put(50,50){\LARGE{\textbf{?}}}
\end{picture}
\end{center}


\begin{defin}[Perception-Action Cycle]
The cycle \[\emph{environment} \rightarrow \emph{sensor} \rightarrow \emph{decision} \rightarrow \emph{action} \rightarrow \emph{environment} \dots\]
\end{defin}

\section{Applications of Intelligent Agents} %------------------------------------------------

\begin{itemize_packed}

\item \textbf{Finance}
\begin{itemize}
\item The environment is the market.
\item The sensors indicate the prices.
\item The actuators make trades.
\end{itemize}

\item \textbf{Robotics}
\begin{itemize}
\item The environment is the physical world.
\item The sensors are cameras, microphones, tactile sensors, etc.
\item The actuators are motors, wheels, speakers, etc.
\end{itemize}

\item \textbf{Games}
\begin{enumerate}
\item Chess
\begin{itemize}
\item The environment is the other player (i.e., you).
\item The sensors indicate your moves.
\item The actuators are the robot's moves.
\end{itemize}
\item Video games: the goal is to make the computer player seem "real."
\end{enumerate}

\item \textbf{Medicine}
\newline Called a "Diagnostic Agent." Takes data (the symptoms) from the patient, but gives its analysis to the doctor.

\item \textbf{The World Wide Web}
\newline A Webcrawler.

\begin{center}
\begin{picture}(200,100)
\thicklines
\put(20,20){\line(1,0){70}}
\put(90,20){\line(0,1){70}}
\put(20,20){\line(0,1){70}}
\put(20,90){\line(1,0){70}}
\put(38,92){Crawler}
\put(170,55){\oval(40,90)}
\put(167,74){\begin{rotate}{-90}\large{WWW}\end{rotate}}
\put(158,90){\vector(-4,-1){80}}
\put(100,80){\begin{rotate}{15}Web pages\end{rotate}}

\put(30,40){\usebox{\DB}}
\put(120,0){\usebox{\stickman}}

\put(80,40){\vector(2,-1){30}}
\put(110,10){\vector(-2,1){30}}
\put(80,14){\begin{rotate}{-30}Query\end{rotate}}
\end{picture}
\end{center}

\end{itemize_packed}

\section{Attributes of Environments} % -------------------------------------------------------

\begin{defin}[Fully Observable vs. Partially Observable]
An environment is \emph{\textbf{fully observable}} if the agent can sense enough data at any time to make the optimal decision. It is \emph{\textbf{partially observable}} if not enough information is immediately detectable (via the sensors) to make the optimal decision.
\end{defin}

\textbf{Example:} In poker, not all of the cards are visible. The agent must use memory and/or logic to attempt to guess them.

\begin{defin}[Deterministic vs. Stochastic]
An environment is \emph{\textbf{deterministic}} if the agent's actions uniquely determine the outcome. It is \emph{\textbf{stochastic}} if the outcome of an action is not totally predictable.
\end{defin}

\textbf{Example:} Throwing dice is stochastic, because it involves randomness.

\begin{defin}[Discrete vs. Continuous]
An environment is \emph{\textbf{discrete}} if there are finitely-many action choices, and finitely-many things that can be sensed. It is \emph{\textbf{continuous}} if there are infinitely-many.
\end{defin}

\begin{defin}[Benign vs. Adversarial]
An environment is \emph{\textbf{benign}} if the environment has no objective that would contradict that of the agent. It is \emph{\textbf{adversarial}} if the environment counteracts what the agent is trying to achieve.
\end{defin}

\textbf{Example:} The weather is benign. Chess is adversarial.

%===============================================================================

\part{Problem Solving}

\begin{defin}[Problem]
A problem is defined by the following functions, defined on the sets \textbf{S} of all possible states and \textbf{A} of all possible actions:
\begin{itemize_packed}
\item \emph{Actions} : \textbf{S} $\rightarrow$ \textbf{A} gives all possible actions at the given state.
\item \emph{Result} : (\textbf{S} $\times$ \textbf{A}) $\rightarrow$ \textbf{S} gives the resulting state of applying an action to the given state.
\item \emph{GoalTest} : \textbf{S} $\rightarrow$ \emph{\{true, false\}} indicates if a given state is a goal.
\item \emph{PathCost} : ($s_1\xrightarrow{a_1}s_2\xrightarrow{a_2}s_3$...) $\rightarrow$ $\mathbb{Z}$ indicates the cost of a series of actions. Note that if \emph{PathCost} is additive, then it can be defined simply by \emph{StepCost} : (\textbf{S} $\times$ \textbf{A}) $\rightarrow$ $\mathbb{Z}$, which gives the cost for a single action.
\end{itemize_packed}
\end{defin}

\begin{defin}[Frontier]
The farthest parts that have been \emph{explored} (but not visited!).
\end{defin}

\section{Search algorithms} %-----------------------------------------------------------------

The following algorithm will be studied in this chapter:

\begin{algorithm}
\caption{TreeSearch}
\begin{algorithmic}[1]
\REQUIRE problem
\ENSURE TreeSearch = (Path to goal) or (fail)
\STATE frontier = \{[initial state]\}
\LOOP
	\IF{frontier is empty}
		\RETURN fail
	\ENDIF
	\STATE \label{remove_choice} path = remove\_choice(frontier)
	\STATE s = path.end
	\IF{GoalTest(s)}
		\RETURN path
	\ENDIF
	\FOR{a in Actions(s)}
		\STATE \label{addResult} add Result(s,a) to frontier
	\ENDFOR
\ENDLOOP
\end{algorithmic}
\end{algorithm}

Before going further, an optimization is immediately apparent. The \textbf{GraphSearch} algorithm improves on \textbf{TreeSearch} by eliminating redundant paths (i.e., backtracking). It modifies TreeSearch in the following ways:
\begin{itemize_packed}
\item An extra variable, \emph{explored}, maintains a set of all nodes visited.
\item After calling \emph{remove\_choice} on line \ref{remove_choice}, add \emph{s} to \emph{explored}
\item In the \textbf{for}-loop, do not execute line \ref{addResult} if $Result(s,a) \in explored$ or $frontier$
\end{itemize_packed}

Different choices for how to define the \emph{remove\_choice} function produce different ways to search the tree. We have:

\begin{defin}[Breadth-first search]
\emph{remove\_choice} always returns the state with the \underline{shortest path}.
\end{defin}

\begin{defin}[Uniform-cost aka "cheapest-first" search]
\emph{remove\_choice} always returns the state with the \underline{lowest total cost}.
\end{defin}

\begin{defin}[Depth-first search]
\emph{remove\_choice} always returns the state with the \underline{longest path}.
\end{defin}

Note that:

\begin{itemize_packed}
\item Breadth-first search is optimal, in that it always finds the \textbf{shortest path}.
\item Uniform-cost search is optimal, in that it always finds the \textbf{cheapest path} (assuming that all step costs are strictly positive).
\item Depth-first search is not optimal, but has the advantage of needing less storage space than the others.
\end{itemize_packed}

\begin{defin}[complete]
A \emph{complete} algorithm will always find a solution.
\end{defin}

To improve on the preceding search algorithms, more information is needed. Suppose we have available a heuristic function $h$ : \textbf{S}$\rightarrow\mathbb{R}^+$ that estimates the distance from any state to the goal.

\begin{defin}[Greedy best-first search]
\emph{remove\_choice} always returns the state \underline{closest to the goal}.
\end{defin}

\begin{defin}[A* search]
\emph{remove\_choice} always returns the state with the \underline{minimum value of $f$}. The function $f$ is defined as $f = g + h$, where $g = PathCost$.
\end{defin}

The A* algorithm will find the lowest-cost path \textbf{IF} the distance function $h$ satisfies the condition:
\begin{center}
$h(s)$ $\leq$ the true cost from $s$.
\end{center}
In this case, we say that $h$
\begin{itemize_packed}
\item never over-estimates
\item is \textbf{optimistic}
\item is \textbf{admissible}
\end{itemize_packed}

An admissible heuristic can be derived from a formal statement of a problem by relaxing one or more of the constraints.

\section{Summary} %-----------------------------------------------------------

The techniques of problem-solving discussed in this chapter can be applied when the problem being studied is:
\begin{itemize_packed}
\item fully observable
\item discrete
\item deterministic
\item static (i.e., only the agent can change the state of the environment)
\item the set of all possible actions at any given state is known.
\end{itemize_packed}

\section{A note on implementation} %----------------------------------------------------------

A state is implemented on a computer as a data structure called a \textbf{node}. A node consists of four fields:
\begin{itemize_packed}
\item \textbf{State}: a description of the state of the environment at the end of the path.
\item \textbf{Action}: the action taken to arrive at that state.
\item \textbf{Cost}: the total cost of getting to that state.
\item \textbf{Parent}: a pointer to the previous node in the path
\end{itemize_packed}

\noindent\textbf{Example}: Suppose we have the path
\begin{center}
$A \rightarrow S \rightarrow F$.
\end{center}
This would be implemented with the following nodes:

\begin{center}
\begin{picture}(250,100)
\put(3,49){\textbf{state}}
\put(3,37){\textbf{action}}
\put(3,25){\textbf{cost}}
\put(3,13){\textbf{parent}}

\put(50,10){\usebox{\nodeBox}}
\put(53,49){A}
\put(53,37){$\emptyset$}
\put(53,25){0}
\put(53,13){$\emptyset$}

\put(130,10){\usebox{\nodeBox}}
\put(133,49){S}
\put(133,37){A$\rightarrow$S}
\put(133,25){...}
\put(133,13){A}

\put(210,10){\usebox{\nodeBox}}
\put(213,49){F}
\put(213,37){S$\rightarrow$F}
\put(213,25){...}
\put(213,13){S}

\put(212,16){\vector(-4,1){50}}
\put(132,16){\vector(-4,1){50}}

\end{picture}
\end{center}

In addition,
\begin{itemize_packed}
\item The frontier nodes are best stored as two data structures:
\begin{itemize_packed}
\item A \textbf{priority queue}, since we need to be able to dynamically add new nodes and remove the "best"
\item A \textbf{set}, since we need to be able to check membership.
\end{itemize_packed}
\item The explored nodes are best stored as a \textbf{set}, since we need to dynamically add new nodes and check membership.
\end{itemize_packed}

\part{Probability in AI}


\end{document}